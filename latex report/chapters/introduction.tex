No matter the method of diagnosis, an early, accurate diagnosis can be the difference between life and death.\newline
\\
Hundreds of years ago, doctors could only diagnose patients based on symptoms or by invasive surgery, but today, over fifty percent of medical diagnoses are performed using radiological imaging modalities \cite{1}. With the growing quantity and quality of radiological methods, radiologists have had to deal with an ever-increasing workload. Radiologists often analyze hundreds of images per week. Errors are both possible and dangerous.\newline
\\
In the 1950’s, the first Artificial Intelligence based medical systems modelled the relationship between symptoms and diseases \cite{2}. They used heuristic reasoning to apply knowledge to the available data and rank alternatives continuously, making the most efficient decision at each branching step to approximate the exact solution.\newline
\\
The initial difficulties were complex as scientists and doctors realized machines would have to master intuitive thinking. The art of diagnosis was not monochrome, the machines would have to diagnose multiple concurrent disorders, requiring deep causal reasoning. Additionally, the uncertainties had to be traced and presented clearly. Furthermore, communications between practised radiologists and computer scientists were sometimes ambiguous. It did not help that experts struggled to identify their logical systems \cite{2}. \newline
\\
Today, computer aided diagnosis (CAD) is a concept that requires the collaboration of computer vision and machine learning to teach computers to analyze radiological images the same way a radiologist would. The computer is then able to provide a second opinion for the radiologist or classify most images where a full time radiologist is not affordable \cite{3}.


