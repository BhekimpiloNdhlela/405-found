Artificial Intelligence is a field of study and applied research where the goal is to create a machine that exhibits human-like intelligence \cite{12}. Artificial Intelligence does not simply involve the absorption and regurgitation of large quantities of information. The machine is expected to demonstrate the ability to learn from past experience. It should master knowledge representation and apply reasoning. Finally the machine must display both intuitive and abstract thinking.\newline 
\\
Machine learning is only a subsection of Artificial Intelligence \cite{13}. It is the use of algorithms that allow computers to learn by analyzing data and making statistical predictions and decisions based on data. Computers are able to learn by error correcting.\newline
\\
For our purposes we consider machine learning algorithms called classifiers. Classifiers can be trained using large quantities of real images or by learning to search for features in an image.\newline
\\
The large quantity of data used to train a classifier is called training data. The data usually consists of images, and information about what is in the images as well as a classification. The algorithm learns to classify data based on features in the images \cite{13}. However, these features can be ambiguous at times, multiple diseases could have similar symptoms or the images could simply be unclear.\newline
\\
In computer science, it is said that the computer tries to make a decision boundary, which is based on features and may not necessarily be a straight line, or exist in only two dimensions. Ultimately, the algorithm must make as many correct classifications as possible and minimise its error \cite{13}. 
There are seven steps in the machine learning process. They are gathering data, data preparation, choosing a model, training, evaluation, hyper-parameter tuning and prediction \cite{14}.\newline
\\
At the first step, gathering data, the quality and quantity of data will determine the performance of the classifier \cite{14}. It is important to have a variety of images for the machine to learn on so that it will be prepared to classify correctly regardless of variations.\newline
\\
For data preparation, the data is accumulated and it undergoes certain processes such as randomization, augmentation, normalization and error correction. This is where data imbalances are managed. A collection of data with significantly more healthy than diseased images will train a classifier to be biased toward classifying an image as healthy. The data must be split into training data and evaluation data. It is important for the classifier to be evaluated on data that it was not presented with during training. The ratio of training data to test data depends on the amount of data. The larger the amount of data, the smaller the required ratio of training to test data \cite{14}.\newline
\\
The model chosen depends on what the classifier will be classifying such as images, sounds or numbers. It also depends on the number of features that the classifier has to be able to take into account \cite{13}. The classifier must be able to handle the number of parameters that must be considered in order to determine whether the image should be classified as, for example healthy, pneumonia or tuberculosis. \newline
\\
In the course of training the classifier improves, learning one iteration at a time. Like a radiologist learning on the job, its initial classifications may not be very accurate, but as it undergoes more iterations it learns to classify correctly, improving its performance with experience. Each attempt to improve performance is called a training step \cite{14}. A training step can be done thousands of times to reach the desired accuracy. \newline
\\
Evaluation data is used during evaluation to test the classifier on data it has never seen before. Once it has classified this data, an analyst would have to evaluate the results to assess the performance of the classifier.
If the classifier’s performance was not up to standard, the parameters must be tuned to improve the training \cite{14}. For example, if too many diseased lungs were classified as healthy, perhaps a larger variety of, or more, images of diseased lungs is needed during training. Other improvements include an increased number of times that the model is trained on the same data, an increased learning rate (how much the model is adjusted at each training step) or a change in the initial conditions \cite{14}. The hyper-parameter tuning is more of an art than a science, it depends on the model’s function and where it is lacking.\newline
\\
Finally, the model is able to predict and classify the chest x-ray images, diagnosing patients. Hereafter the accuracy, specificity, stability, F1 score and more can be calculated, reported and compared to other state of the art models or human radiologists.\newline
\\
The goal is to build a classifier that is as good as or better than the average human radiologist so that it can assist radiologists or perform the job of a standard radiologist in areas where a radiologist is needed but not financially attainable.

